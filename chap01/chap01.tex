% 打开时默认折叠书签
\hypersetup{bookmarksopen=false}

\part{总论}
%\chapter*{总论}
\markboth{总论}{总论}

\begin{figure}[htbp]
	\centering
	\includegraphics[width=1.0\linewidth]{chap01/fig_1_0}
	\caption{人脑的白质纤维结构,其中显示了胼胝体和脑干通路。
		该图像根据\textit{核磁共振成像}数据和扩散光谱成像技术构建,该技术使用水分子的扩散速率和偏好方向在核磁共振图像中产生对比度,以揭示在纤维束中行进的轴突束。
		纤维按方向进行颜色编码:
		左右方向为X轴的红色;前后方向为Y轴的绿色;
		上下方向为Z轴的蓝色
		(由南加州大学神经成像实验室和哈佛大学生物医学成像中心提供的来自\href{www.humanconnetomproject.org}{人类连接体项目联合会}的连接体扫描仪数据集)。}
	\label{fig:1_0}
\end{figure}

在 20 世纪下半叶,生物学的中心焦点是基因。
现在,在 21 世纪上半叶,焦点已经转移到神经科学,特别是\textit{心理生物学}。
我们希望了解感知、行为、学习和记忆的过程。
一个只有 1.5 公斤重的大脑器官是如何构思无限、发现新知识并产生人类思想、情感和行为的非凡个性?
这些非凡的心理能力如何在器官内分布?
什么规则将一个区域的解剖组织和细胞生理与其在心理状态中的特定作用联系起来?
基因对行为有什么影响?
神经细胞中的基因表达是如何受到发育和学习过程的调节?
经验是如何改变大脑处理后续事件的方式?
以及这种处理在多大程度上是无意识的?
最后,神经疾病和精神疾病的神经基础是什么?
在《神经科学原理》的介绍部分,我们开始讨论这些问题。
在这样做的过程中,我们描述了神经科学如何试图将神经回路的计算逻辑与大脑联系起来,即定义的神经回路中的神经细胞活动如何介导复杂的心理过程。


在过去的几十年里,技术进步为大脑的科学研究开辟了新的视野。
今天,有可能将神经元互连回路的细胞动力学与大脑中感知和运动行为的内部表示联系起来,并将这些内部机制与可观察的行为联系起来。
新的成像技术使我们能够可视化人类大脑的活动,即识别大脑中与特定思维和感觉模式及其相互联系模式相关的特定区域。


% In the first part of this book
在本书的第一部分中,我们考虑了心理功能在多大程度上可以定位到大脑的特定区域。
我们还研究了这些功能在多大程度上可以从单个神经细胞的特性、它们的分子成分和突触连接来理解。
在本书的后面部分,我们详细研究了大脑认知和情感功能的基本机制:感知、行为、动机、情绪、学习和记忆。


人脑是一个由 800 多亿个神经细胞组成的网络,这些神经细胞在系统中相互连接(神经回路)构建我们对外部世界的感知、集中我们的注意力、指导我们的决策、并实施我们的行为。
因此,理解大脑的第一步是了解神经元是如何组织成信号通路,以及它们是如何通过突触传递进行交流。
我们将在本书中提出的主要观点之一是:在\textit{发育过程中建立}并在\textit{经验过程中完善}的\textit{突触连接特异性}是行为的基础。
我们还必须了解行为的先天决定因素和环境决定因素,在这些决定因素中,基因编码的蛋白质最初控制着神经回路的发育,然后可以通过基因表达的经验依赖性变化来改变神经回路。


% A new science of mind is emerging through
通过将现代细胞和分子生物学技术、大脑成像、理论和临床观察应用于认知、情绪和行为的研究,一门新的心理科学正在兴起。
神经科学强化了希波克拉底在两千多年前首次提出的观点,即对心智的正确研究始于对大脑的研究。
认知心理学和精神分析理论强调了人类心理体验的多样性和复杂性。
这些学科现在可以通过从神经科学中深入了解大脑功能来丰富。
未来的任务是对心理过程进行研究,以实证神经科学为基础,关注心理的内部表示和状态是如何产生的问题。


% Our goal is to provide not simply the facts
我们的目标不仅仅是提供事实,而是提供大脑组织、功能和计算的原理。
神经科学原理并没有将人类思想的复杂性简化为一组分子或数学公理。
相反,它让我们能够在大脑的复杂性中欣赏到某种美(即达尔文式的优雅),这种复杂性解释了思维和行为。
人们可能会问,从更基本的神经机制的详细解剖中收集到的一个想法是否包含了对更高大脑功能的见解?
简单反射的组织是否与手的意向性运动有关?
在发育中的脊髓中建立回路的机制是否与存储记忆的机制有关?
% 原文:Are the neural processes that awaken us from sleep similar to those that allow an unconscious process to pierce our conscious awareness?
% 翻译:将我们从睡眠中唤醒的神经过程与允许无意识过程刺穿我们意识的神经过程相似吗?
% TODO: 说反了?让我们的意识刺穿无意识过程?
\textit{将我们从睡眠中唤醒}的神经过程是否与\textit{让意识刺穿无意识}的神经过程相似?
我们希望读者在深入研究其事实基础时,会对这些原理感到高兴。
毫无疑问,这是一项正在进行的工作。





\chapter{大脑和行为} \label{chap:chap1}
生物科学的最后前沿和终极挑战是理解意识的生物学基础,以及感知、行为、学习和记忆的大脑过程。
在过去的几十年里,生物科学内部的显著一致为应对这一巨大挑战奠定了基础。
对基因进行测序并推断它们编码蛋白质的氨基酸序列揭示了神经系统中的蛋白质与身体其他部位的蛋白质之间意想不到的相似性。 
因此,建立细胞功能的总体规划成为可能,该规划为包括细胞神经科学在内的所有细胞生物学提供了一个共同的概念框架。


当前生物学统一的挑战是心理学(心智科学)和神经科学(大脑科学)的统一。
在这种统一的方法中,不将心灵和身体视为独立的实体,而是据此将所有行为都视为大脑功能的结果。
我们通常所说的心智是由大脑执行的一系列操作。 
大脑过程不仅是行走和进食等简单运动行为的基础,也是典型人类的所有复杂认知和行为的基础(即思考、说话和创作艺术品)。
作为推论,所有表征精神疾病的行为障碍(感觉上的情感障碍和思维上的认知障碍)都是脑功能紊乱的结果。


大脑中数十亿个独立的神经细胞如何产生行为和认知状态,以及这些细胞如何受到环境(包括社会经验)的影响? 
将行为用大脑活动来解释是神经科学的重要使命,神经科学在这一领域的进展构成了本书的核心主题。


神经科学必须不断面对某些基本问题。
为了理解思维过程、肢体动作或动作意愿,我们需要在生物学描述上达到怎样的适当层次? 
在某些神经疾病状态下,为何动作会显得流畅或生硬,甚至无意中发生? 
这些问题的答案可能源自观察神经细胞中\textit{脱氧核糖核酸}表达的模式,以及这些模式如何调控神经元的电学特性。
然而,我们还需要了解由特定大脑区域中的许多神经元组成的神经回路以及许多大脑区域中特定回路的活动如何协调。


% Is there a level of biological description
是否存在一个最适合的生物学描述层级?
简短的回答是:这取决于具体情况。
如果一个人的目标是了解和治疗某些遗传性癫痫病,那么\textit{脱氧核糖核酸}测序和单个神经元电特性的测量可能足以产生有效的治疗方法。
而如果研究者关注的是学习、感知和探索等认知功能,那么很可能需要对神经回路的系统性连接以及不同大脑区域之间的交互作用进行分析。


现代神经科学的目标是将这些专业的描述水平整合为一个连贯的科学体系。
这种努力促使我们必须面对一系列新问题。
如果心理过程能够被定位到大脑的特定区域,那么这些区域的功能与它们的解剖结构及生理特性之间存在怎样的关系?
是否需要一种类型的神经回路来处理视觉信息,另一种类型的神经回路来解析语言,再有其他类型的神经回路来一步步移动?
或者,不同功能的神经回路是否遵循某些共同的组织原则?
对于必要的神经计算,我们是否应该将其理解为基于单个神经元或神经元群体所表示信息的操作?
信息是否通过单个神经细胞的电活动来表示,还是说它是分布在整个神经网络中,以至于任何一个细胞所携带的信息并不比计算机内存中的随机位比特元更具信息量?
正如我们将要看到的,关于组织水平、细胞特化和功能定位的问题在整个神经科学中不断重现。


为了阐述这些观点,我们将探讨现代神经科学如何描述语言(语言是人类一种独特的认知行为)。
在这样做时,我们将广泛关注大脑皮层(人类大脑中发育最完善的部分)的运作。
将看到大脑皮层如何组织成功能不同的区域(每个区域由许多神经元组成),以及如何根据特定区域内特定组互连神经元的活动来分析高度复杂行为的神经组织。
第~\ref{chap:chap3}~章描述了简单反射行为的神经回路如何在细胞水平上运作,说明了感觉信号和运动信号的相互作用如何引发运动行为。



% 结构(几何)和功能(行为)
\section{关于大脑与行为之间的关系,提出了 2 种相反的观点}

我们对神经细胞、大脑和行为的理解是在20世纪通过对 5 种实验传统的综合而形成:解剖学、胚胎学、生理学、药理学和心理学。


% 解剖学
公元 2 世纪的希腊医生\textit{盖伦}提出:神经是传递大脑和脊髓分泌的液体到身体其他部位的通道。
他的观点在西方医学中占据主导地位,直到显微镜揭示了神经组织中细胞的真实结构。
即便如此,神经组织直到19世纪末才成为一门独立科学领域的研究对象,当时意大利人\textit{卡米洛$\cdot$高尔基}和西班牙人\textit{圣地亚哥$\cdot$拉蒙$\cdot$卡哈尔}对神经细胞进行了详细、准确的描述,但他们对大脑功能的工作原理得出了 2 个截然不同的结论。


\textit{高尔基}开发了一种用银盐对神经元进行染色的方法,这种方法可以在显微镜下显示出它们的整个细胞结构。
基于这些研究,\textit{高尔基}得出结论:神经细胞不是相互隔离的独立细胞,而是在一个连续的组织或合胞体网络中共同作用。
利用\textit{高尔基}的染色技术,\textit{卡哈尔}观察到每个神经元通常都有一个细胞体和 2 种类型的突起:一端是分支树突,另一端是长长的电缆状轴突。
\textit{卡哈尔}据此得出结论:神经组织不是\textit{合胞体},而是离散细胞网络。
在这项工作的过程中,\textit{卡哈尔}发展了神经元学说的一些关键概念和许多早期证据,即个体神经元是神经系统的基本构建块和信号元件的原则。


% In the 1920s the American embryologist
% 胚胎学
在 1920 年,美国\textbf{胚胎学}家\textit{罗斯$\cdot$哈里逊}发现:树突和轴突是从细胞体中生长出来的,即使在组织培养中每个神经元都与其他神经元隔离开来也是如此。
\textit{哈里逊}还证实了\textit{卡哈尔}的主张,即轴突的尖端会产生扩张(即生长锥),它将发育中的轴突引导至其目标(即其他神经细胞或肌肉)。
卡哈尔和哈里逊的这两项发现都为神经元学说提供了强有力的支持。
随着电子显微镜的引入,神经元学说的最终确凿证据出现在 20 世纪 50 年代中期。
\textit{桑福德$\cdot$帕莱}的一项具有里程碑意义的研究明确地证明了突触的存在:突触是神经细胞的特殊区域,允许它们之间进行化学信号或电信号传递。


% 生理学
神经系统的\textbf{生理学}研究始于 18 世纪后期,当时意大利医师和物理学\textit{路易吉$\cdot$加尔瓦尼}发现肌肉和神经细胞会产生电能。
现代电生理学起源于 19 世纪的 3 位德国生理学家:\textit{约翰内斯$\cdot$米勒}、\textit{埃米尔$\cdot$杜$\cdot$博伊斯-雷蒙德}和\textit{赫尔曼$\cdot$冯$\cdot$亥姆霍兹},他们成功地测量了神经细胞轴突上电活动的传导速度,并进一步表明:
一个神经细胞的电活动以可预测的方式影响相邻细胞的活动。


% 药理学
\textbf{药理学}在 19 世纪末对神经系统和行为的理解产生了第一次影响,当时法国的\textit{克劳德$\cdot$伯纳德}、德国的\textit{保罗$\cdot$埃尔利希}和英国的\textit{约翰$\cdot$兰利}证明药物不会随机作用于细胞,而是通常与位于细胞膜中的离散受体结合。
这一发现揭示了神经细胞可以通过化学物质进行相互通讯。


% 心理学
关于行为的\textbf{心理学}思考可以追溯到西方科学的开端,当时古希腊哲学家推测行为的原因以及心灵与大脑的关系。
在随后的几个世纪里,出现了 2 种主要观点。
到了17世纪,笛卡尔提出了身心二元论,
他认为大脑负责调节感知、动作、记忆、欲望和情感(这些在较低等动物中可以找到的一切)。
但是心智(更高层的人类行为有意识体验特征的心智功能)并不存在大脑或身体的任何其他部分,而是代表一个精神实体的灵魂。
笛卡尔认为,灵魂通过松果体与大脑的机器进行交流(松果体是大脑中线的一个微小结构)。
笛卡尔的立场在现代哲学或神经科学中几乎没有影响力。
事实上,神经科学的基本前提是心智是大脑及其神经活动的产物。
我们这样说并不是为了表明神经科学的目标是通过将心灵还原为生物学成分来否定其存在,而是为了阐释心灵的生物学基础。


% 解剖学
% 局部
% Attempts to join biological and psychological
早在 1800 年,维也纳医生和神经\textbf{解剖学}家\textit{弗朗兹$\cdot$约瑟夫$\cdot$加尔}就开始尝试将生物学和心理学概念结合到行为研究中,当时他提出了一种全新的身心观念。
他主张大脑是心灵的器官,一切精神功能都体现在大脑中。
因此,他拒绝了笛卡尔认为思想和身体是分开实体的观点。
此外,他认为大脑皮层不是一个单一的器官,而是包含许多专门的器官,大脑皮层的特定区域控制着特定的功能。 
如图~\ref{fig:1_1}~所示,\textit{加尔}列举了至少 27 个不同的大脑皮层区域或器官; 
后来又增加了更多,每一种都对应于一种特定的心智能力。
\textit{加尔}将智力过程(例如评估因果关系、计算和感知秩序的能力)分配给了大脑的前部。
浪漫的爱情(恋爱)和好斗的本能特征被分配到大脑后部。
即使是最抽象的人类行为(慷慨、隐秘和虔诚)也被分配到大脑中的一个位置。


\begin{figure}[htbp]
	\centering
	\includegraphics[width=0.55\linewidth]{chap01/fig_1_1}
	\caption{大脑功能定位的早期映射。
		根据 19 世纪的颅相学学说,好斗、灵性、希望和责任心等复杂特征由专门的“器官”控制,大脑皮层的不同区域随着特征的发展而扩大。
		这些大脑局部区域的扩大被认为会在覆盖的头骨上产生特征性的凹凸,从中可以确定一个人的性格。
		这张映射取自 18 世纪初的一幅图画,显示了 42 个智力和情感“器官”。}
	\label{fig:1_1}
\end{figure}


尽管\textit{加尔}的身心统一理论和他关于某些功能局限于特定大脑区域的观点被证明是正确的,但今天的主流观点是:许多高层心理功能很可能是高度分布式的。
此外,\textit{加尔}的局部化实验方法极其幼稚。
\textit{加尔}没有根据经验定位功能,而是通过研究大脑并将心理属性的缺陷与肿瘤或中风后特定区域的病变相关联,\textit{加尔}摒弃了所有来自脑部病变研究的证据(无论是通过临床检查发现的还是在实验动物身上通过手术产生的)。
受\textit{相面术}的影响(这是一门基于面部特征揭示性格的流行科学),\textit{加尔}认为:具有特定认知能力的人头骨上的凸起和脊线可以识别大脑中这些能力的中心。
他假设大脑区域的大小与该区域所代表的心智能力的相对重要性有关。
因此,特定智力的锻炼会导致相应的大脑区域生长,而这种生长又会导致覆盖在上面的头骨突出。


% Gall first had this idea as a young boy when
小时候,\textit{加尔}注意到那些擅长背作业的同学都有突出的眼睛,于是他就有了这个想法。
他得出结论:这是大脑前部与语言记忆有关的区域过度发育的结果。
当他成为一名年轻的医生,并负责维也纳一家精神病院时,他进一步发展了这个想法。
在那里,他开始研究患有偏执狂的患者,这种疾病的特征是对某些关键思想过分感兴趣或强烈渴望从事某些特定行为,比如盗窃、谋杀、色情、极端宗教信仰。
他推断,由于患者在所有其他行为中都表现良好,大脑缺陷一定是离散的,原则上可以通过检查这些患者的头骨来定位。
\textit{加尔}对局部大脑功能的研究催生了颅相学(这是一门根据头骨的详细形状确定人格和性格的学科)。


% 整体
% In the late 1820s, Gall’s ideas
在 1820 年代后期,法国生理学家\textit{皮埃尔$\cdot$弗卢龙}对\textit{加尔}的想法进行了实验分析。 
\textit{弗卢龙}使用实验动物破坏了\textit{加尔}所主张的一些大脑功能中心,进而试图分离出这些“大脑器官”对行为的贡献。
从这些实验中,\textit{弗卢龙}得出结论:特定的大脑区域并不负责特定的行为,而是所有的大脑区域(尤其是前脑的大脑半球)都参与了每一次心理操作。
\textit{弗卢龙}提出,大脑半球的任何部位都对半球的所有功能有所贡献。
因此,大脑半球任何一个区域的损伤都应该平等地影响所有更高的功能。
基于这些发现,\textit{弗卢龙}在1823年写道:“所有的感知、所有的意志在这些(大脑)器官中占据同一位置;
因此,感知、构想和意志的能力本质上构成了一种单一的能力。


% The rapid acceptance of this belief
这种观点后来被称为大脑的\textbf{整体}观,其迅速被接受不仅部分基于\textit{弗卢龙}的实验工作,
同时也代表了一种文化上的反击,反对将人类心智视为一个生物器官的唯物主义观点。
它代表对以下观念的拒绝:灵魂不存在、所有心理过程都可以归结为大脑活动、以及通过锻炼可以提升心智(这些观念对于当时的欧洲宗教权威和贵族阶层来说是不可接受的)。


% 连接
然而,在 19 世纪中叶,法国神经学家\textit{皮埃尔$\cdot$保尔$\cdot$布洛卡}、德国神经学家\textit{卡尔$\cdot$韦尼克}和英国神经学家\textit{休林斯$\cdot$杰克逊}严重挑战了整体观点。
例如,\textit{杰克逊}在对局灶性癫痫(一种以身体特定部位开始抽搐为特征的疾病)的研究中表明:不同的运动功能和感觉功能可以追溯到大脑皮层的特定部位。
\textit{布洛卡}、\textit{韦尼克} 和\textit{杰克逊}的区域研究被\textit{查尔斯$\cdot$谢灵顿}和\textit{卡哈尔}扩展到细胞水平,他们支持一种称为细胞\textbf{连接}主义的大脑功能理论。
根据这一理论,单个神经元是大脑的信号传递单元;
它们按功能组排列,并以精确的方式相互连接。
\textit{韦尼克}和法国神经学家\textit{朱尔斯$\cdot$代热林}的研究表明:不同的行为由不同的相互关联的大脑区域产生。


关于大脑功能的定位,第一个重要的证据来自于研究大脑如何产生语言的过程。
在考虑相关的临床和解剖学研究之前,首先要回顾一下大脑的整体结构,包括它的主要解剖区域。
这需要我们定义一些神经解剖学家用来描述大脑和脊髓部分之间三维空间关系的基本导航术语,
这些术语在文本框~\ref{box:1_1}~和图~\ref{fig:1_2}~中介绍。


\begin{proposition}[神经解剖学导航术语] \label{box:1_1}
	
	\quad \quad 如图~\ref{fig:1_2}~所示,中枢神经系统各组成部分在体内的位置和方向参照 3 个轴进行描述:嘴侧-尾侧轴、背侧-腹侧轴和内侧-外侧轴。
	这些术语使神经解剖学家能够描述大脑和脊髓部分之间的空间关系。
	它们有助于在同一物种的个体发育或疾病情况下对其大脑进行比较。
	例如,它们也有助于比较不同动物物种的大脑,以了解大脑的进化。

\end{proposition}


\begin{figure}[htbp]
	\centering
	\includegraphics[width=0.95\linewidth]{chap01/fig_1_2}
	\caption{中枢神经系统沿着 3 个主要轴进行描述\cite{martin2012neuroanatomy}。
		A. 嘴侧表示朝向鼻子和尾侧朝向尾巴。
		背侧是指朝向动物的背部,腹侧是指朝向腹部。
		在低等哺乳动物中,这 2 个轴的方向在发育到成年生活的过程中一直保持不变。
		在人类和其他高等灵长类动物中,纵轴在脑干中弯曲大约 110 度。
		由于这种挠曲,相同的位置术语在指代挠曲下方和上方的结构时具有不同的含义。
		在弯曲下方,在脊髓中,嘴侧意味着朝向头部,尾侧意味着朝向尾骨(脊柱的下端),腹侧(前侧)意味着朝向腹部,背侧(后侧)意味着朝向背部。
		在弯曲上方,嘴侧意味着朝向鼻子,尾侧意味着朝向后脑勺,腹侧意味着朝向下巴,背侧意味着朝向头顶。
		术语上位通常与背侧同义,下位与腹侧相同。
		B. 内侧意味着朝向大脑中部,外侧意味着朝向外侧面。
		C. 当对大脑进行切片分析时,切片通常是在 3 个基本平面之一中制作的:水平、冠状或矢状。}
	\label{fig:1_2}
\end{figure}



\section{大脑具有不同的功能区域}

中枢神经系统是一个 2 个半球且基本对称的结构,包含 2 个主要部分:脊髓和大脑。
如文本框~\ref{box:1_2}~和图~\ref{fig:1_3}~所示,脑包括 6 个主要结构:延髓、脑桥、小脑、中脑、间脑和大脑。
这些结构中每一个都包含具有特定连接方式和发育起源的明显不同的神经元群。
在延髓、脑桥、中脑和间脑中,神经元通常分组为称为核的不同的簇。
大脑和小脑的表面由一大片折叠的神经元组成,分别称为大脑皮层和小脑皮层,其中神经元以固定的连接模式分层组织。
如图~\ref{fig:1_4}~所示,大脑还包含许多位于皮层下方的结构,包括基底神经节和杏仁核。


% 使用tcolorbox可以控制斜体和粗体,
% proposition模板中的楷体没有粗体和斜体
% elegentboox.cls:914: fontupper=\citshape
% \begin{tcolorbox}[title={中枢神经系统的解剖学组织}]
% \end{tcolorbox}
\begin{proposition}[中枢神经系统的解剖学组织] \label{box:1_2}
	% \textbf{}只会加粗数字7
	中枢神经系统有 7 个主要部分
	
	\quad \quad \textit{脊髓}是中枢神经系统的最尾部,接收和处理来自四肢和躯干的皮肤、关节和肌肉的感觉信息,并控制四肢和躯干的运动。
	如图~\ref{fig:1_3}A~所示,它被细分为颈椎、胸椎、腰椎和骶椎区域。
	
	\quad \quad 脊髓向前端延伸形成\textit{脑干},脑干由延髓、脑桥和中脑组成。
	脑干接收来自头部皮肤和头部肌肉的感觉信息,并为头部肌肉组织提供运动控制。
	脑干还将信息从脊髓传递到大脑,也可以将信息从大脑传递到脊髓,并通过网状结构调节觉醒水平和意识水平。
	
	\quad \quad 脑干包含几个细胞体集合,即脑神经核。
	其中一些细胞核接收来自头部皮肤和肌肉的信息;
	另一些核控制面部、颈部和眼睛肌肉的运动输出。
	还有一些核专门处理来自 3 种特殊感官的信息:听觉、平衡觉和味觉。
	
	\quad \quad \textit{延髓}直接位于脊髓的嘴侧,包括几个负责重要自主功能的中心,如消化、呼吸和心率的控制。
	
	\quad \quad \textit{脑桥}(从嘴侧到\textit{延髓})传递从大脑半球到小脑的运动信息。
	
	\quad \quad 脑桥后面的\textit{小脑}调节运动的力量和范围,并参与运动技能的学习。
	它在功能上与脑干的 3 个主要器官相连:延髓、脑桥和中脑。
	
	\quad \quad \textit{中脑}位于脑桥的嘴侧,控制着许多感觉和运动功能,包括眼球运动以及视觉和听觉反射的协调。
	
	\quad \quad \textit{间脑}位于中脑的嘴侧,包含 2 个结构。
	丘脑处理从中枢神经系统其他部分到达大脑皮层的大部分信息。
	下丘脑调节自主神经功能、内分泌功能和内脏功能。
	
	\quad \quad \textit{大脑}由 2 个大脑半球组成,每个半球由褶皱严重的外层(大脑皮层)和 3 个深层结构(基底神经节、海马体和杏仁核的组成部分)组成。
	基底神经节包括尾状核、壳核和苍白球,调节运动执行、运动和习惯学习,这 2 种记忆形式被称为内隐记忆;
	海马体对储存人、地点、事物和事件的记忆至关重要,这种记忆形式被称为外显记忆;
	杏仁核协调情绪状态的自主神经和内分泌反应,包括对威胁的记忆(这是另一种形式的内隐记忆)。
	
	\quad \quad 如图~\ref{fig:1_3}~所示,每个大脑半球分为 4 个不同的叶:额叶、顶叶、枕叶和颞叶。
	尽管皮层区域都高度互联,这些脑叶与不同的功能有关,可以参与广泛的大脑功能。
	枕叶接收视觉信息,对视觉的各个方面都至关重要。
	来自枕叶的信息通过 2 个主要通路进行处理。
	连接枕叶和顶叶的背侧流与视觉空间中物体的位置和操作有关。
	连接枕叶和颞叶的腹侧流与物体身份有关,包括对单个人脸的识别。
	颞叶对处理听觉信息也很重要(它还包含隐藏在其表面下的海马体和杏仁核)。
	额叶与所有皮层区域紧密相连,对高层认知处理和运动规划很重要。
	
	\quad \quad 大约 2/3 的皮层位于脑沟中,许多脑回被覆盖的皮层叶所掩埋。
	如图~\ref{fig:1_4}~所示,通过分离大脑半球以显示大脑的内侧表面,并在死后对大脑进行切片,例如在尸检中,可以看到大脑皮层的完整范围。
	如图~\ref{fig:1_5}~所示,这些信息中的大部分可以通过现代大脑成像在活体大脑中可视化(第~\ref{chap:chap6}~章)。
	这些观点也提供了白质和皮层下灰质的观点。
	
	\quad \quad 大脑皮层表面看不到的 2 个重要区域包括扣带皮层和岛叶皮层。
	扣带皮层位于胼胝体的背面,对情绪、疼痛感知和认知的调节很重要。
	岛叶皮层埋在额叶、顶叶和颞叶的里内,在情绪、稳态和味觉感知中发挥着重要作用。
	这些内部视图还提供了对\textit{胼胝体}的检查,\textit{胼胝体}是连接 2 个半球的突出轴突\textit{纤维束}。
	
	\quad \quad 上述不同的大脑区域通常分为 3 个更广泛的区域:后脑(包括延髓、脑桥和小脑);
	中脑(包括顶盖、黑质、网状结构和中脑导水管周围灰质);
	和前脑(包括间脑和大脑)。
	中脑和后脑(减去小脑)包括与脑干相同的结构。
	神经系统的解剖组织在第~\ref{chap:chap4}~章中有更详细的描述。
		
\end{proposition}


\begin{figure}[htbp]
	\centering
	\includegraphics[width=1.0\linewidth]{chap01/fig_1_3}
	\caption{中枢神经系统的划分。
		A. 中枢神经系统可分为 7 个主要区域,从最尾部的脊髓区域,到脑干(延髓、脑桥和中脑),到间脑(包括丘脑和下丘脑),到端脑或大脑(大脑皮层、底层白质、皮层下核和基底神经节)。
		B. 大脑的 4 个主要叶以覆盖它们的颅骨部分命名。
		这张大脑的侧视图仅显示左侧大脑半球。
		中央沟将额叶和顶叶分开。
		外侧沟将额叶与颞叶分开。
		初级运动皮层占据紧靠中央沟头端的回。
		初级体感皮层占据中央沟尾部的回。
		C. 在右半球的内侧视图中,当半球分开时,可以看到大脑的进一步分裂。
		\textit{胼胝体}包含一大束连接 2 个半球的轴突。
		扣带皮层是大脑皮层的一部分,围绕着大脑皮层。
		初级视觉皮层占据距状沟。}
	\label{fig:1_3}
\end{figure}


\begin{figure}[htbp]
	\centering
	\includegraphics[width=1.0\linewidth]{chap01/fig_1_4}
	\caption{在死后组织的脑切片图中可以看到大脑半球的主要皮层下和深部皮层区域。
		4 个连续的冠状切片(A–D)沿着大脑侧视图(右上角,插图)上指示的延髓-尾轴制作。
		基底神经节包括尾状核、壳核、苍白球、黑质和丘脑底核(未显示)。
		丘脑将感觉信息从外围传递到大脑皮层。
		杏仁核和海马体是埋藏在颞叶内的大脑皮层区域,对情绪反应和记忆很重要。
		脑室包含并产生脑脊液,脑脊液浸润脑沟、脑池和脊髓\cite{nieuwenhuys2007human}。}
	\label{fig:1_4}
\end{figure}


现代脑成像技术可以看到活人中这些结构的活动(见第~\ref{chap:chap6}~章)。
当人们在受控条件下从事特定任务时,脑成像通常用于评估大脑离散区域的代谢活动。
这些研究提供的证据表明,特定类型的行为比其他行为更能激发大脑特定区域的活动。
如图~\ref{fig:1_5}~所示,大脑成像生动地表明认知操作主要依赖于大脑皮层,即覆盖 2 个大脑半球的皱纹灰质。


\begin{figure}[htbp]
	\centering
	\includegraphics[width=1.0\linewidth]{chap01/fig_1_5}
	\caption{可以在活人的大脑中成像主要皮层和皮层下区域。
		A. 这张示意图显示了大脑的主要表面和深部区域,包括脊髓的延髓末端,以供参考。
		B. 在 A 部分绘制的主要大脑分区在活人脑的磁共振图像中很明显。}
	\label{fig:1_5}
\end{figure}


如图~\ref{fig:1_3}B~所示,在大脑的每个半球中,覆盖其上的皮层被划分为 4 个叶,这些叶的名称来源于覆盖它们的颅骨:额叶、顶叶、枕叶和颞叶。
每个叶都有几个特征性的深折叠,这是一种将一大片皮层包装到有限空间中的进化策略。
这些回旋的顶部称为脑回,中间的沟称为脑沟或裂隙。
人与人之间非常相似的更突出的脑回和脑沟具有特定的名称。
例如,如图~\ref{fig:1_3}B~所示,中央沟将中央前回(一个与运动功能有关的区域)与中央后回(一个处理感觉功能的区域)分开。
如第~\ref{chap:chap6}~章所述,无论是在死后组织中(图~\ref{fig:1_4}),还是实际上使用磁共振成像(图~\ref{fig:1_5}),几个突出的脑回仅在 2 个半球之间的内侧表面可见(图~\ref{fig:1_3}C),其他脑回位于脑裂和脑沟深处,因此只有在大脑被切片时才可见。


每个大脑叶都有其专门的功能。
额叶主要负责短期记忆、规划未来的行为以及运动控制;
顶叶负责调节躯体感觉,构建身体形象,并将其与外部空间联系起来来;
枕叶主要处理视觉信息;
而颞叶则负责处理听觉、识别物体和面孔,此外,通过其深层结构的海马体和杏仁核还参与学习、记忆和情感的处理。


大脑皮层的组织具有 2 个重要特征。
首先,每个半球主要关注身体对侧的感觉过程和运动过程。
因此,来自身体左侧的感觉信息在传递至大脑皮层之前,会在脊髓中交叉到神经系统的右侧。
同样,右半球的运动区控制着身体左半边的运动。
第 2 个特征是 2 个半球虽然外观相似,但在结构或功能上并不完全对称。



\section{认知能力局部化的第一个有力证据来自语言障碍研究}


大脑皮层中首先被认定与认知功能密切相关的区域是那些与语言处理有关的区域。
这些发现来自对失语症的研究,失语症是一种语言障碍,最常发生在大脑组织的某些区域因中风、供应大脑半球一部分的血管闭塞或破裂而受损时。
在19世纪后半期,失语症研究领域迅速连续取得了许多重要的发现。
综合来看,这些科学进步构成了神经科学研究人类行为的历史上最为激动人心且意义重大的篇章之一。


法国神经学家\textit{皮埃尔$\cdot$保尔$\cdot$布洛卡}是首位确认大脑中与语言处理相关的特定区域的学者。
\textit{布洛卡}受到\textit{加尔}试图绘制大脑高级功能图的影响,但他并没有将行为与头骨上的隆起相联系,而是将临床失语症的证据与死后发现的大脑损伤相联系。
在1861 年,他写道:“我曾认为,如果有一门颅相学,那将是(大脑皮层中)脑回的颅相学,而不是头上隆起的颅相学。” 
基于这种洞察力,布洛卡创立了神经心理学,这是一门心理过程的经验科学,他将其与加尔的颅相学区分开来。


同年,布洛卡描述了一位名叫\textit{勒博涅}的病人,该病人因中风而失去说话能力,尽管他能够很好地理解语言。
这位患者没有舌头、嘴巴或声带的运动障碍,这些障碍可能会影响他的说话能力。
实际上,他能够轻松地说出单词、吹口哨和唱出旋律。
但他无法说出符合语法的句子,也无法通过书写来表达思想。
如图~\ref{fig:1_6}~所示,对这位病人去世后大脑的检查发现,左前额叶的后下方区域存在损伤,现在这个区域被称为布洛卡区。
\textit{布洛卡}研究了 8 名相似的患者,均在该区域有病变,并且每个病例的病变都位于左侧大脑半球。
这一发现促使\textit{布洛卡}在 1864 年宣布:“我们用左半球说话!” 。


\begin{figure}[htbp]
	\centering
	\includegraphics[width=0.7\linewidth]{chap01/fig_1_6}
	\caption{语言处理涉及左脑半球的几个区域。
		布洛卡区控制言语的产生。
		它位于控制形成单词的嘴巴和舌头运动的运动区附近。
		韦尼克区处理语言的听觉输入,对理解言语很重要。
		它位于初级听觉皮层和角回附近。
		法国神经学家\textit{朱尔斯$\cdot$代热林}在 1890 年代提出,角回中的多模态感觉区域整合了来自视觉和听觉的信息来表示单词,但最近的研究表明更多的腹侧枕颞皮层区域用于处理视觉单词。
		\textit{韦尼克}区通过双向通路与\textit{布洛卡}区相通,其中一部分由弓状束组成\cite{geschwind1979specializations}。}
	\label{fig:1_6}
\end{figure}


\textit{布洛卡}的工作激发了对与特定行为相关的大脑皮层区域的探索,很快这一探索就取得了成果。
1870 年,\textit{古斯塔夫$\cdot$弗里奇}和\textit{爱德华$\cdot$希茨格}通过电刺激中央前回的特定区域,成功地在狗身上引发了特征性的肢体动作(如伸出爪子),这一发现在科学界引起了轰动。
这些区域总是位于对侧运动皮层。
因此,最常用于书写和熟练动作的右手是由左半球控制的,左半球也控制着语言功能。
因此,在大多数人中,左半球被认为是主导半球。



1876年,26岁的\textit{卡尔$\cdot$韦尼克}克发表了一篇现在被广泛认为是经典的论文:《失语症的症状群:基于解剖学的心理研究》。
在这篇论文中,他描述了失语症的另一种类型,这是一种理解上的障碍,而非言语表达上的障碍:是一种接受性失语,与布洛卡研究的表达性失语相反。
布洛卡的病人能够理解语言但无法说话,而韦尼克的病人能够说出单词,但无法理解语言,他们会产生毫无意义但语法正确的句子。
此外,这种新型失语症的病变位置与布洛卡描述的区域不同。
如图~\ref{fig:1_6}~所示,病变发生在大脑皮层后部,颞叶与顶叶相接的地方。



基于这一发现以及\textit{布洛卡}、\textit{弗里奇}和\textit{希茨格}的研究,\textit{韦尼克}制定了一种语言的神经模型,试图整合并扩展当时关于大脑功能的主导理论。
颅相学家和细胞连接主义者认为大脑皮层是由功能特定的区域组成的马赛克,而整体论者和场理论者则认为每个心理活动都涉及整个大脑皮层。
韦尼克提出,只有最基本的心理功能(那些与简单的知觉和运动活动有关的)是由大脑皮层中离散的局部区域的神经元独立介导的。
他认为,更复杂的认知功能是由几个功能区域之间的相互连接产生的。
通过在连接主义框架内整合局部功能的原则,韦尼克强调了一个单一行为的不同组成部分可能在大脑的多个区域中被处理的观点。
因此,他是第一个提出分布式处理概念的人,这一概念现在是神经科学的一个核心原则。




\textit{韦尼克}假设语言涉及独立的运动和感觉程序,每个程序都由皮层的不同区域控制。
如图~\ref{fig:1_6}~所示,他提出,控制言语嘴部运动的运动程序位于\textit{布洛卡区},这个区域恰好位于控制嘴、舌、上颚和声带的运动区前方。
接着,他将控制单词感知的感觉程序归因于他所发现的颞叶区域,现在这个区域被称为韦尼克区。
韦尼克区被听觉皮层以及现在统称为联合皮层的区域所包围,这些区域整合了听觉、视觉和躯体感觉。
根据\textit{韦尼克}的模型,这 2 个语言中心之间的交流是通过一个名为弓状束的大型轴突束进行的。



因此,韦尼克制定了第一个连贯的语言神经模型,该模型在第~\ref{chap:chap55}~章中进行了重要的修改和阐述,至今仍具有重要意义。
根据这个模型,口语或书面语的神经处理始于专门处理听觉或视觉信息的皮层的单独感觉区域。
然后,这些信息通过中间的联合区域传递,这些区域提取出适合识别口语或书面语的特征,将信息传递到韦尼克区,在那儿信息被识别为语言,并与意义相联系。




韦尼克模型的强大之处不仅在于其完整性,还在于其预测能力。
该模型正确预测了第 3 种类型的失语症,即由连接中断引起的传导性失语。
在这种失语症中,语言的接收区和表达区本身并未受损,但连接这 2 个区域的神经纤维(即弓状束)遭到破坏。
现在被称为传导性失语的这种失语症,其特征是频繁的基于声音的言语错误(音素性错语)、重复困难,以及言语工作记忆的严重限制。
患有传导性失语的患者能够理解他们听到和阅读的单词,并且在说话时没有运动障碍。
然而,他们无法连贯地说话;他们可能会省略单词的一部分或替换错误的声音,在逐字重复他们听到、读到或从记忆中回忆起的多音节单词、短语或句子时遇到极大困难。
尽管他们痛苦地意识到自己的错误,但他们连续尝试自我纠正却往往不成功。





在韦尼克的成果部分启发下,由解剖学家\textit{科比尼安$\cdot$布罗德曼}领导,20 世纪初在德国兴起了一种新的皮层定位学派,该学派根据细胞形状及其层状排列的变化来区分皮层的功能区域。
如图~\ref{fig:1_7}~所示,布罗德曼运用这种细胞构造方法,区分了人类大脑皮层中 52 个解剖学和功能上各有不同的区域。




\begin{figure}[htbp]
	\centering
	\includegraphics[width=0.65\linewidth]{chap01/fig_1_7}
	\caption{20 世纪初,人类大脑皮层被分为 52 个独立的功能区。
		显示的区域是由解剖学家布罗德曼根据独特的神经细胞结构和细胞层的特征排列确定的。
		该方案至今仍在广泛使用,并不断更新。
		布罗德曼定义的几个区域被发现可以控制特定的大脑功能。
		例如,区域 4 是运动皮层,负责\textit{自主运动}。
		区域 1、2 和 3 构成初级体感皮层,主要从皮肤和关节接收感觉信息。
		第 17 区是初级视觉皮层,它接收来自眼睛的感觉信号并将它们传递到其他区域以进行进一步处理。
		区域 41 和 42 构成初级听觉皮层。
		该图仅显示了皮层外表面的可见区域。}
	\label{fig:1_7}
\end{figure}

尽管有关大脑皮层功能区域离散性的生物学证据十分有力,但直到20世纪50年代,整体论对大脑的观念仍然主导着实验性思维和临床实践。
这种出人意料的状况很大程度上归功于几位支持整体论的著名神经科学家,包括英国神经学家\textit{亨利$\cdot$海德}、俄国行为生理学家\textit{伊万$\cdot$巴甫洛夫}和美国心理学家\textit{卡尔$\cdot$拉什利}。



最有影响力的是\textit{拉什利},他对细胞构造方法用于皮层功能映射持深度怀疑态度。
“‘理想’的构造图几乎毫无价值,”\textit{拉什利}写道。
“区域划分在很大程度上在解剖学上没有意义,并且对于推测皮层的功能划分具有误导性。” 
他对不同脑损伤对老鼠走迷宫学习能力的影响的研究进一步加深了他的怀疑。
从这些研究中,\textit{拉什利}得出结论,学习缺陷的严重程度取决于损伤的大小,而不是其确切位置。
失望的\textit{拉什利}(以及他之后的许多其他心理学家)得出结论,学习和其他高级心理功能在大脑中没有特定的位置,因此不能归因于特定的神经元集合。




基于自己的观察,\textit{拉什利}通过进一步最小化单个神经元、特定神经连接、甚至特定大脑区域在产生特定行为中的作用,重新表述了\textit{整合场}的观点。
根据\textit{拉什利}的大规模作用理论,对于记忆等函数至关重要的是大脑的整体质量,而非其区域组成部分。
\textit{拉什利}的老鼠实验现在被重新诠释。
各种研究表明,\textit{拉什利}使用的迷宫学习不适合寻找局部皮层功能,因为它涉及太多的运动能力和感觉能力。
即使被剥夺了一种感觉能力(比如视觉),老鼠仍然可以依靠触觉或嗅觉来学习走迷宫。
此外,正如我们将在本书后面看到的,许多心理功能是由不止一个区域或神经元通路介导的。
因此,一个给定的功能可能不会被单个损伤消除。
这一点在考虑大脑的认知功能时尤为重要。
例如,空间知识的掌握得到许多顶叶联合区域的支持,这些区域将视觉与潜在的眼球转动、头部转动、手的伸展等动作联系起来。
原则上,这些联合区域中的任何一个都可以弥补另一个区域的损伤。
要产生明显的空间知识缺陷(空间失认症)(见第~\ref{chap:chap59}~章),需要对顶叶造成严重的损伤。
这样的观察结果似乎支持了大规模作用理论,但我们现在已经认识到,它与包含功能冗余概念的功能定位是相容的。


很快,关于大脑功能定位的证据变得无可辩驳。
从 1930 年代后期开始,英国的\textit{埃德加$\cdot$阿德里安}和美国的\textit{韦德$\cdot$马歇尔}和\textit{菲利普$\cdot$巴德}发现,触摸猫身体不同部位会在大脑皮层的特定区域引发电活动。
通过系统地探测身体表面,他们在\textit{布罗德曼}描述的大脑皮层特定区域,建立了身体表面的精确映射。
这一结果表明,可以根据解剖学标准(如细胞类型、细胞层次、细胞连接,以及最为重要的行为功能)明确地定义出功能上有区别的皮层区域。
正如我们在后续章节中将看到的,功能特化是大脑皮层的一个关键组织原则,甚至延伸到皮层区域内的单个细胞柱。
实际上,大脑被划分为比布罗德曼所设想的更多的功能区域。



% More refined methods made it possible to learn
更为精细的方法使我们能够更深入地了解参与语言的大脑不同区域的功能。
在 20 世纪 50 年代后期,\textit{怀尔德$\cdot$潘菲尔德}和后来的\textit{乔治$\cdot$奥杰曼}重新研究了对产生语言至关重要的皮层区域。
在对癫痫患者进行脑部手术并局部麻醉期间,要求觉醒的患者在用小电极刺激暴露的不同皮层区域时说出物体的名称(或以其他方式使用语言)。
如果皮层的某个区域对语言至关重要,电刺激的应用就会阻碍患者说出物体的名称的能力。
通过这种方式,\textit{潘菲尔德}和\textit{奥杰曼}得以确认布洛卡和韦尼克描述的大脑皮层的语言区域。
正如我们将在第~\ref{chap:chap55}~章中了解到的那样,语言的神经网络远远超出了布洛卡和韦尼克所描述的范围,更为广泛和复杂。


最初,我们对语言解剖组织的所有了解几乎都来自对脑损伤患者的研究。
如今,\textit{功能性核磁共振成像}和其他非侵入性方法允许我们在健康人阅读、说话和思考时进行分析(第~\ref{chap:chap6}~章)。
\textit{功能性核磁共振成像}不仅证实了阅读和说话激活了不同的大脑区域,还揭示了在没有感觉输入的情况下仅仅思考 1 个词的含义会激活左前额叶皮层的 1 个不同区域。
实际上,即使在传统语言区域内,各个子区域也会根据我们思考单词、表达它们以及从其他单词的排列(即句法)中解析它们含义的方式在不同程度上被激活。
新的成像工具不仅承诺告诉我们哪些区域参与其中,还揭示了它们相互连接的功能逻辑。



现代方法论带来的一个巨大惊喜是,如此多的大脑皮层区域在语言理解和产生过程中被激活。
这些包括左半球的传统语言区域,由\textit{布洛卡}、\textit{韦尼克}和\textit{代热林}确定; 它们在右半球的同源区域;以及新识别的区域。
功能成像倾向于阐明不同募集的区域,而中风、肿瘤或损伤造成的损伤则区分出对一个或多个功能至关重要的大脑区域。
因此,如图~\ref{fig:1_6}~所示,曾经被认为专门用于语言产生的布洛卡区实际上也参与了包括理解在内的多种语言任务。
在某些情况下,功能成像需要对病变研究确定的关键区域进行细化或修正。
例如,如图~\ref{fig:1_6}~所示,阅读现在被认为除了募集顶叶皮层的角回之外,还会募集腹侧枕颞皮层的专门区域。





因此,大脑中的语言处理不仅体现了局部功能的原则,还体现了这一原则的更复杂阐述,即许多具有专门功能的不同神经结构属于系统的一部分。
也许这是关于局部化和分布式过程争议的自然和解(即少数不同的区域)每个区域都与一组特定的功能相关,并通过它们的相互作用为感知、行为和思维的现象学做出贡献。
大脑对任务的处理可能与我们的直觉告诉我们的不同。
谁会猜到,对一个物体的运动和颜色的神经分析会在不同的通路中进行,而不是通过单一的通路来形成一个统一的物体感知?
同样,我们可能会期待,语言的神经组织可能不会完全符合普遍语法理论的公理,但却支持语言学理论所描述的无缝功能。




对脑损伤患者的研究持续为我们理解大脑如何组织语言提供了重要的见解。
其中一个最引人注目的发现来自于对聋人的研究,这些人在遭受脑部损伤后失去了使用手语(例如,\textit{英国手语}或\textit{美国手语})进行交流的能力。
手语通过手部动作而非发声来传达信息,并通过视觉而非听觉来感知,但它们的结构复杂性与口语相同。
与口语处理一样,手语处理也定位于左半球。左半球的损伤会对手语产生特定的影响,就像对口语一样,可能会影响手语理解(在韦尼克区受损后)、语法或流畅性(在布洛卡区受损后)。
这些临床观察得到了功能性神经影像学的支持。
不出所料,如图~\ref{fig:1_8}~所示,手语和口语的产生和理解虽然不涉及完全相同的大脑区域,但两者之间的重叠确实非常显著。
甚至有证据表明,处理手语的构成部分(例如,使用的手形)会涉及到一些在对语音进行押韵判断时相同的大脑区域。



\begin{figure}[htbp]
	\centering
	\includegraphics[width=0.82\linewidth]{chap01/fig_1_8}
	\caption{手语聋人和听力正常的人共享共同的语言处理区域。
		大脑皮层中涉及口语或手语识别的区域,通过\textit{功能性核磁共振成像}识别。
		黄色高亮显示左右大脑半球(分别为左列和右列)在理解语言时比在执行感知任务时更活跃的区域。
		对于失聪的手语者(上排),突出显示的区域在理解英国手语期间比在检测叠加在同一静止手语者身上的视觉刺激期间更加活跃。
		对于有听力的说话者(下排),在理解视听语音期间突出显示的区域比在观看静止(无声)说话者时检测音调期间更活跃\cite{macsweeney2002neural}。}
	\label{fig:1_8}
\end{figure}


这些观察说明了 3 个要点。
首先,语言处理主要发生在左半球,这一点与处理语言所使用的感觉和运动通路无关。
其次,听觉输入并不是左半球语言能力出现和运作的必要条件。
第三,口语只是左半球所调节的一系列语言技能中的一种。



对其他行为的研究为大脑拥有不同认知系统的观点提供了额外的支持。
这些研究表明,复杂的信息处理需要许多相互连接的皮层和皮层下区域,每个区域都专注于处理感觉刺激或运动的特定方面。
例如,对物体位置、大小和形状的感知意识依赖于众多顶叶联合区域的活动,这些区域将视觉与潜在的动作联系起来,如移动眼睛、调整头部方向、伸手和调整手的形状以抓握。
顶叶区域并不直接启动这些动作,而是评估感觉信息作为这些潜在动作的证据。
它们从\textit{背侧视觉通路}接收信息(有时称为\textit{空间通路},但更恰当地称为\textit{方法通路})以构建对物体位置和其他空间属性的认知状态。
腹侧视觉通路(或者说是\textit{内容通路})也关注可能的动作,但这些动作与社交和觅食相关。
这些关联建立了对物体、面孔、食物和潜在配偶可取性的认知。
从这个意义上说,\textit{内容通路}也可能是\textit{方法通路}。



\section{心理过程是大脑中基本处理单元之间相互作用的产物}

大脑功能定位的证据在过去经常被拒绝,这有多种原因。
颅相学家以夸张的形式引入了大脑功能定位的概念,但没有充分的证据支持。
他们将大脑皮层的每个区域想象成一个独立的精神器官,专门负责人格的一个完整而独特的方面,就像胰腺和肝脏作为独立的消化器官一样。
\textit{弗卢龙}对颅相学的否定,以及随后的\textit{整合场}观点(反对定位)与细胞连接主义者(支持定位)之间的辩论,是对一个过于简单且缺乏充分实验证据的理论的回应。


在韦尼克发现大脑中语言的模块化组织(具有独特功能的互连节点)之后,我们现在认为所有认知能力都是分布在大脑多个区域的许多处理机制相互作用的结果。
也就是说,特定的大脑区域并不完全负责特定的心智能力,而是作为基本的处理单元共同发挥作用。
感知、运动、语言、思想和记忆都是通过这些区域内离散大脑区域(计算模块)的串行和并行处理的相互联系而成为可能。
因此,正如许多早期神经学家所认为的那样,单个区域的损伤不一定会导致认知功能(或能力)的完全丧失。
即使某种行为最初消失了,它也可能随着大脑未受损部分重新组织它们的连接而部分恢复。
此外,当\textit{局灶性损伤}对心理功能产生不利影响时,它可能会通过破坏其他主要位点的功能(精神分裂症)而间接影响。
实际上,这种观察使韦尼克的学生\textit{科特$\cdot$戈德斯坦}采纳了更为整体的观点。



因此,将心理功能视为严格由一系列神经细胞和大脑区域(每个都直接连接到下一个)来调节是不准确的,因为在这种安排下,当单个连接受损时,整个过程就会被打断。
一个更现实的比喻是,一个过程由模块网络中的多个并行通路组成,这些通路相互作用,并最终汇聚到一组共同的目标上。
网络中单个通路的故障可能会影响该通路携带的信息,而不会破坏整个系统。
网络的其余部分可能可以调整其性能以适应某一通路的故障。



大脑中的模块化处理之所以被接受得较慢,是因为直到最近,很难展示心理操作的哪些组成部分是由特定的通路或大脑区域介导的。
也不容易以能够产生可检验假设的方式来定义心理操作。
然而,随着现代认知心理学和大脑科学在近几十年的不断融合,我们开始意识到心理功能可以成功地分解为子功能。


为了说明这一点,请考虑我们如何学习、存储和回忆有关物体、人物和事件的信息。
简单的内省表明,我们将每一条知识存储为一个单一的表征,可以通过记忆触发的刺激甚至仅通过想象力来回忆。
例如,你对苹果所知的一切似乎都存储在一个完整的表征中,无论你看到一个特定的苹果、苹果的一部分、红色或绿色的苹果、单词“苹果”,还是关于发现重力的虚构故事,都可以访问。
然而,我们的经验并不准确地指导我们了解知识是如何在记忆中存储的。





关于苹果的知识并不是作为一个单一的连贯表征存储的,而是被细分为不同的类别并分别存储。
大脑的一个区域存储了你拿苹果的方式、你摸苹果时的感觉(与新鲜度有关)、颜色(与偏好或新鲜度相关)、你如何向他人传达苹果的存在或味道,以及它与计算机、物理学家、蠕虫、蛇和圣经中的伊甸园等的语义关联。
概念“苹果”包含了所有这些考虑因素以及更多。
一个自然的假设是,一个包含许多细节的连贯概念必须存在于大脑的一个单一位置;然而,同样合理的假设是,像“苹果”这样的统一概念在心智中以多种神经结构之间的多重链接的形式存在,每个结构都包含一种特定类型的信息,并通过记忆检索的作用进行协调。


心理过程模块化组织最令人惊讶的例子是,我们对自我意识(一个有自我意识的存在,我们所说的“我”的总和)是通过连接我们 2 个大脑半球中独立的回路来实现的,每个回路都调节着它自己的意识感。
\textit{罗杰$\cdot$斯佩里}、\textit{迈克尔$\cdot$加扎尼加}和\textit{约瑟夫$\cdot$伯根}在研究那些因为治疗癫痫而切断了胼胝体(连接 2 个大脑半球的主要神经束)的患者过程中,做出了一个非凡的发现,即意识甚至不是一个单一的过程。


他们发现每个半球都有自己的意识,能够独立于另一半球运作。
因此,当一位患者用左手阅读他最喜欢的书时,控制左手但对语言理解只起很小作用的右半球发现,它从简单地看书获得的原始视觉信息是无聊的。
右半球命令左手放下书。
另一位患者会用左手穿衣服,同时用右手脱衣服。
每个半球都有自己的思想!
此外,主导半球有时会对非主导半球的表现发表评论,经常对它无法知道解决方案的问题表现出一种错误的信心,而解决方案却只提供给了非主导半球。




这些发现将曾经属于哲学和精神分析领域的意识研究带入了神经科学的范畴。
正如我们将在后面的章节中看到的,本章中描述的许多问题在意识的神经理论中再次出现。
没有人质疑很多信息处理(也许是最大的份额)没有达到有意识的认识这一想法。
没有人会质疑这样一个观点:大量的信息处理(也许是最主要的部分)并没有达到意识层面。
当感官信息、行为规划或想法确实变成意识时,神经科学试图解释这种转变的中介机制。
尽管目前还没有令人满意的解释,一些脑科学家会将这一过程比作注意力焦点的转移,由不同的神经元群体介导,而另一些人则认为意识需要在大脑广泛分离的区域之间发生功能互动质的变化。




我们花了这么长时间才理解哪些心理活动是由大脑的哪些区域介导的,主要原因是我们正在处理生物学中最深奥的谜题:解释意识和自我意识的神经机制。
目前还没有令人满意的理论来解释为什么只有部分到达我们眼睛的信息会导致对某个物品、人物或场景的主观意识状态。
我们知道,我们只意识到我们心理思考的一小部分,而那些确实突破意识层面的想法,必须源自大脑无意识执行的步骤。
正如我们在第~\ref{chap:chap56}~章中提出的那样,一些意识之谜的答案可能比想象的更接近。


同时,我们对神经科学的理解上存在的当前差距也带来了实际的认识论挑战。
在我们对知觉、行为和认知的描述中,我们不得不依赖于我们对世界、身体和思维的有意识体验。
然而,这样做的同时,我们有可能错误地描述了许多并未进入意识层面的心理过程。
例如,我们倾向于用与感官信息的主观体验一致的术语来描述知觉问题,而即使是复杂但无意识的对知觉内容的了解,可能在行为效用(即提供的可能性)方面与实际行为更为相似,实际上就是在回答“这是我可能会选择去吃、坐在上面或进一步探索的东西吗?”
同样,认知过程,如推理、策略制定和决策,很可能是以一种与我们从有意识思考中推断出的步骤只是大致相似的方式由大脑执行的。





这些警示说明有一个明显的推论。
许多认知功能在无意识的情况下发生的这一洞察,提出了神经科学在研究更基本行为时揭示的原则可能为理解更复杂的认知过程提供见解的可能性。
通过对被训练执行复杂任务的动物大脑进行神经记录,我们对决策、推理、规划和注意力分配等认知过程有了更深入的理解。
这些实验模型经常推广到人类功能,而在它们不足之处,它们激发了新的假设。
通常情况下,即使我们从理解上的空白中得不到深入的见解,也能获得灵感。


为了分析大脑如何产生特定的心理过程,我们不仅必须确定过程的哪些方面依赖于大脑的哪些区域,还必须了解相关信息是如何被表示、路由和转换的。
现代神经科学试图在许多不同的尺度上整合这样的理解。
例如,对单个神经细胞及其分子成分的研究阐明了电兴奋性和突触连接背后的机制。
对细胞和简单回路的研究有助于深入了解神经计算,从控制净激励等基本操作到更复杂的计算(例如从原始感官数据中提取有意义的信息)。
研究不同回路和大脑区域之间的相互作用可以解释我们如何协调分散的肌肉群或表达对一个命题的信念。
所有这些层次的知识都通过数学形式化、计算机模拟和心理学理论相互联系在一起。
这些概念工具现在可以与现代生理技术和大脑成像方法结合起来,使我们能够追踪活体动物和人类心理过程的实时发展。
事实上,今天神经科学令人激动源于这样一种信念:我们掌握了人类思想和行为的生物学原理,并且可能很快被用来阐明和改善人类状况。






\section{亮点}

1. 神经科学致力于在多个组织层次上理解大脑,这些层次包括从细胞及其组成部分到心智运作的整个过程。


2.神经科学的基础原理跨越了时间、复杂性和状态的不同层面,涵盖了从\textit{细胞}到\textit{动作}和\textit{思维},从发展过程中的专业知识的\textit{学习}和\textit{遗忘},以及从正常功能到神经缺陷和恢复的各个方面。
作为第一步,人们必须了解神经\textit{细胞}的电特性及其与其他神经细胞的\textit{连接},以及神经系统从支持细胞到\textit{通路}的组织。


3. 神经元学说认为:单个神经细胞(神经元)是神经系统的基本构建块和信号传递元素。


4. 神经元被组织成具有专门功能的回路。
最简单的回路负责调节反射;而更复杂的认知功能则需要更为复杂的回路。
这种组织原则将神经元学说扩展到了细胞连接主义。


5. 即便在复杂的回路中,也可以识别出关键节点,这些节点是与特定功能相关联的区域。
大脑功能定位的第一个确凿证据来自于对语言产生特定障碍的研究。


6、2 个大脑半球接收来自身体对侧的信息,并控制对侧身体的行为。


7. 虽然大脑功能定位原理优于其主要的历史替代理论(\textit{整合场理论}和\textit{整体活动理论}),但它仍在不断完善之中。
大脑皮层的任何区域都不是独立于其他皮层和皮层下结构而发挥作用的。


8. 功能模块化组织原则是对定位原则的一个主要改进。
大脑中包含许多信息表征,这些表征既根据特定特征对某些计算的相关性进行组织,也根据这些信息可能被用于的各种目的进行组织。
这是一种关于目的或潜在行为的冗余形式。


9. 脑科学的未来将需要整合跨越传统学科界限的思想。
我们必须对各种资源保持开放的心态,以指导我们的直觉和研究策略,从\textit{意识的崇高本质}到看似平凡的\textit{全身麻醉对丘脑周围细胞环中钙传感器的影响}。



















